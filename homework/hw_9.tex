\documentclass{article}
\title{Taller \#9. F\'isica Computacional / FISI 2025 \\Semestre
  2013-I. \\ Profesor: Jaime E. Forero Romero} 
\date{Abril 30 2013}
\begin{document}
\maketitle

{\bf Esta tarea debe resolverse por parejas (i.e. grupos de 2
  personas) y debe estar en un repositorio de la cuenta de github de
  uno de los miembros de cada equipo con un commit final hecho antes del
  medio d\'ia del martes 4 de Mayo del 2013}  

El objetivo de este taller es usar exploraci\'on MCMC y estad\'istica
bayesiana para resolver un problema que nace de una historia de la vida real
(!). Es un problema planteado por Ivan Caicedo, estudiante de
maestr\'ia en Uniandes. 

En una regi\'on del espacio marcada por coordenadas $x_{0}, y_{0},
z_{0}$ hay una fuente de part\'iculas. A diferentes
posiciones $z_1,z_2$ est\'an ubicados dos detectores que
registran la traza dejada por una part\'icula. Estos detectores son
planos y est\'an alineados para ser perpendiculares al plano $x-y$.  

La traza de cada part\'icula est\'a descrita entonces por las
posiciones dejadas en cada detector. De esta manera cada traza est\'a
descrita por 6 n\'umeros: $x_1,y_1, z_1, x_2,y_2, z_2$. Nosotros
contamos con las mediciones de diferentes trazas, el problema es
encontrar los valores m\'as probables de $x_0,y_0,z_0$ para el origen de estas trazas.


En el archivo \verb"homework/hw9_data/traces.dat" encontrar\'a un archivo de texto con 6
columnas. Cada columna corresponde a los n\'umeros que describen las
trazas y cada fila corresponde a una traza diferente.
\begin{enumerate}
\item 
Escriba un programa en $python$ que use un me\'etodo MCMC para
encontrar la posici\'on m\'as probable de $x_0,y_0,z_0$. 

\item 
Escriba un programa que preparare tres gr\'aficas de densidad de
probabilidad (pi\'ensenlo como histogramas bidimensionales
adecuadamente normalizados) en los planos $x-y$, $x-z$, $y-z$ para el
origen m\'as probable.   
\end{enumerate}

En la calificaci\'on se dar\'a un 50\% a cada uno de los puntos del 1
al 2. Solamente se recibir\'an tareas que est\'en en un repositorio de
github. 

Enviar un email a {\tt  j.e.forero.romero} en {\tt gmail.com} con el
subject


\verb"RESPUESTA TALLER 9 FISICA COMPUTACIONAL". En el cuerpo del texto
debe ir la direcci\'on del repositorio donde est\'a la tarea.  


\end{document}
