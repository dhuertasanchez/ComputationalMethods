\documentclass{article}
\usepackage{hyperref}
\textheight=25.5cm
\textwidth=16.0cm
\oddsidemargin=-0.5cm
\topmargin=-3.0cm
\usepackage[pdftex]{graphicx}
\usepackage[utf8]{inputenc}
\usepackage[spanish]{babel}
 \usepackage{amsmath}
\title{Taller \#7 de M\'etodos Computacionales\\ FISI 2028, Semestre 2014 - 20}
\author{Profesor: Jaime Forero}
\date{Viernes 31 de Octubre, 2014}
\begin{document}
\maketitle
\thispagestyle{empty}


{\bf Importante}
\begin{itemize}

\item Todo el c\'odigo fuente y los datos se debe encontrar en un
  repositorio en github con un commit final hecho antes del medio
  d\'ia del martes 11 de noviembre. El nombre del repositorio debe ser
  \verb"Apellidos1_Apellidos2_hw7", por ejemplo si trabajo con
  Nicol\'as deber\'iamos crear un repositorio llamado
  \verb"Forero_Garavito_hw7".  

\item 
  La nota m\'axima de este taller es de 100 puntos. Se otorgan 1/3
  de los puntos si el c\'odigo fuente es razonable, 1/3 si se puede
  compilar/ejecutar y 1/3 si da los resultados correctos.  

\item
  El miercoles 5 de noviembre habr\'a una sesi\'on para presentar
  avances de diferentes grupos en cada una de las preguntas del
  taller. Hay un bono de 15 puntos para cada grupo que presente un
  avance significativo.  


\end{itemize}


\begin{enumerate}

\item {\bf Cuerda Vibrando} (50 puntos)

(Ver cap\'itulo 18.2 de Landau, Paez, Bordeianu, {\it Soluci\'on de la
  ecuaci\'on de onda hiperb\'olica})\\

Vamos a considerar una cuerda de longitud $L$ descrita por la funci\'on
$u(x,t)$ que corresponde al desplazamiento con respecto a su
posici\'on de equilibrio. Despu\'es de una perturbaci\'on inicial, la
evoluci\'on de $u$ est\'a dada por 

\begin{equation}
\frac{\partial^2 u (x,t)}{\partial x^2} =
\frac{1}{c^2}\frac{\partial^2 u}{\partial t^2}
\end{equation}

donde $c=\sqrt{T/\rho}$ es una velocidad de propagaci\'on con $T$ la
tensi\'on de la cuerda y $\rho$ su densidad \footnote{En clase ya
  resolvimos esta ecuaci\'on con diferentes condiciones iniciales:
  un pulso gaussiano y uno sinusoidal}.

Las condiciones de contorno corresponden a puntos fijos.  Como
condici\'on inicial se tiene que la cuerda est\'a estirada de forma triangular, con el m\'aximo ubicado a $8/10$ de la longitud
total de la cuerda con una altura $1$.  Es decir,

\begin{equation}
u(x,t=0) = 
\begin{cases}
1.25x/L & x \leq 0.8L\\
5-5x/L & x>0.8L
\end{cases}
\end{equation}

Integre esta ecuaci\'on tomando $T=40$N y $L=100$m para
$0<t<120$s. 

El programa debe estar escrito en C y debe poder ejecutarse como
\verb"./string.x rho" donde \verb"rho" es la densidad lineal de la
cuerda. Para hacer pruebas pueden tomar $\rho=0.01$kg/m. 

Este programa debe producir un archivo llamado \verb"string_rho.dat"
donde se guarda la evoluci\'on del sistema en 121 filas (cada fila
correspondiente a la evolucion para el tiempo en segundos
correspondiente, i.e. la fila 20 es la evoluci\'on para $t=19$s) y 101
columnas (cada columna tiene el valor de $y$ para el $x$
correspondiente, i.e. la columna 99 guarda el valor de $u$ para $x=98$m).
Debe existir otro programa en python que toma como argumento
el archivo de datos para producir una gr\'afica
\verb"cuerda_rho.pdf" del mismo tipo de la Figura 18.3 en el libro de
Landau, Paez, Bordeianuu. Todo debe poder ejecutarse con un Makefile. 



\item {\bf Shock}. (50 puntos)

Vamos a trabajar con un fluido unidimensional definido por tres
cantidades: la velocidad $u$, la presi\'on $p$ y la densidad
$\rho$. El gas es\'a descrito por una ecuaci\'on de estado con
exponente adiab\'atico $\gamma=1.4$. Vamos a asumir que este fluido se
puede describir por las ecuaciones de Euler, que no son nada m\'as que
una expresi\'on de la conservaci\'on de masa, momentum y energ\'ia.  

El objetivo de este punto es resolver las ecuaciones de Euler para la
siguiente situaci\'on. Hay un tubo que va desde $x=-10$m hasta $x=10$m
y en la mitad hay una membrana r\'igida. Del lado izquierdo hay un gas es
est\'atico $u_I=0$m/s, tiene densidad alta $\rho_I=1$kg/m$^3$ y una presi\'on
alta $p_I=100$kN/m$^2$. Del lado derecho hay m\'as gas que sigue
siendo est\'atico $u_D=0$m/s pero tienen una densidad baja
$\rho_D=0.125$ kg/m$^3$ y una presi\'on m\'as baja $p_D=10$kN/m$^2$. 

Para $t=0$ la membrana desaparece y el gas es libre de evolucionar
siguiendo las ecuaciones de Euler. El objetivo es calcular la
presi\'on, densidad y velocidad a lo largo del tubo para el tiempo
$t=0.1$s.

¿Cómo van a hacer esto? Siguiendo el esquema num\'erico presentado
aqu\'i:

\url{http://nbviewer.ipython.org/github/numerical-mooc/numerical-mooc/blob/master/lessons/03_wave/03_05_Sods_Shock_Tube.ipynb}.

{\bf No es necesario que lean las lecciones anteriores del MOOC para
  resolver el problema. Basta con que vean con cuidado las ecuaciones
  (8) y (9) que describen el esquema num\'erico necesario para
  evolucionar las variables del sistema.}

El programa debe estar escrito en C y debe poder ejecutarse como
\verb"./sod_test.x t" donde \verb"t" es el tiempo en segundos al cual
se desea evolucionar la soluci\'on. Este programa debe producir un
archivo llamado \verb"estado_t.dat" donde se guarda en cuatro
columnas: la posici\'on a lo largo del tubo, la velocidad, presi\'on y
densidad. Debe existir otro programa en python que toma como argumento
el archivo de datos para producir una gr\'afica
\verb"estado_t.pdf". Todo debe poder ejecutarse con un Makefile.

\end{enumerate}


\end{document}
