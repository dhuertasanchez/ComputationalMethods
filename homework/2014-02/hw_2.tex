\documentclass{article}
\usepackage{hyperref}
\textheight=25.5cm
\textwidth=16.0cm
\oddsidemargin=-0.5cm
\topmargin=-3.0cm
\usepackage[pdftex]{graphicx}
\usepackage[utf8]{inputenc}
\usepackage[spanish]{babel}
\title{Taller \#2 de M\'etodos Computacionales\\ FISI 2028, Semestre 2014 - 20}
\author{Profesor: Jaime Forero}
\date{Viernes 15 de Agosto, 2014}
\begin{document}
\maketitle
\thispagestyle{empty}


{\bf Importante}
\begin{itemize}

\item Los tres archivos de c\'odigo fuente de soluci\'on de esta tarea
  deben subirse a trav\'es de sicuaplus antes de las medio d\'ia del
  viernes 29 de Agosto como un \'unico archivo zip con el nombre
  \verb"NombreApellidos_hw2.zip", por ejemplo yo deber\'ia subir un
  archivo llamado \verb"JaimeForero_hw2.zip" 

\item La nota m\'axima de este taller es de 100 puntos. Se otorgan 1/3
  de los puntos si el c\'odigo fuente es razonable, 1/3 si se puede
  compilar/ejecutar y 1/3 si da los resultados correctos.  
\item 
{\bf Para las personas que entreguen la tarea antes del medio d\'ia
  del viernes 22 de Agosto cada programa se calificar\'a sobre 35
  puntos en lugar de 30. Es decir, pueden sacar 115 puntos en lugar de
  100.} 
\end{itemize}

\begin{enumerate}

\item {\bf Jugando con $\pi$} 

En el Repositorio \verb"ComputationalMethodsData" se encuentra el
archivo \verb"homework/hw_1/Pi_2500000.txt" que contiene cerca de 2.5
millones de d\'igitos del n\'umero $\pi$ en su representaci\'on
decimal. Escriba un programa en C ( el c\'odigo fuente se debe llamar
{\verb"suma_digitos_pi.c"}) que sume los primeros $n$ d\'igitos de la
representaci\'on decimal de $\pi$ . Despu\'es de compilado el programa
debe poder ejecutarse como  

\begin{verbatim}
./a.out n
\end{verbatim}
Donde \verb"n" representa el n\'umero de d\'igitos decimales para sumar. 
\begin{itemize}

\item (5 puntos) El programa debe parar su ejecuci\'on y dar un mensaje de error si $n<0$ o si $n>2500000$.
\item (30 puntos) El programa debe funcionar de manera correcta con valores hasta de $n=2500000$.  El valor final se debe imprimir en pantalla.
\end{itemize}



\item {\bf Volando de una ciudad a otra}

Los aviones que vuelan largas distancias internacionales buscan minimizar su consumo de combustible. Para esto la ruta de vuelo debe ser la m\'as corta entre los puntos de partida y llegada.  El objetivo de este ejercicio es escribir un programa que calcule las coordenadas de esa trayectoria para dos puntos arbitrarios sobre la Tierra.

Para simplificar el problema supondremos que la Tierra es perfectamente esf\'erica. Adem\'as vamos a utilizar los \'angulos $\theta$ (polar) y $\varphi$ (ecuatorial) para describir las coordenadas sobre un punto de la esfera.

Escriba un programa (c\'odigo fuente \verb"trayectoria_esfera.c") que dados dos puntos arbitrarios sobre la esfera (parametrizados por $\theta$ y $\varphi$) calcula la trayectoria m\'as corta que entre esos dos puntos. Despu\'es de compilado el programa debe ejecutarse como

\begin{verbatim}
./a.out theta_1 phi_1 theta_2 phi_2
\end{verbatim}

Donde \verb"theta_1", \verb"phi_1" corresponden a las coordenadas del primer punto y \verb"theta_2", \verb"phi_2" a las del segundo. Ambos valores deben darse en grados.

\begin{itemize}
\item (5 puntos) El programa debe parar su ejecuci\'on y dar un mensaje de error si los \'angulos se encuentran por fuera del rango permitido.
\item (30 puntos) El programa debe escribir en pantalla una serie de $100$ pares de valores de $\theta$ y $\varphi$ que marquen la trayectoria. Estos valores deben corresponder a puntos equidistantes sobre la esfera.
\end{itemize}

\item {\bf Marcha Aleatoria 3D}

En clase hemos ganado experiencia haciendo marchas aleatorias. En este punto vamos a obtener estad\'isticas para una marcha aleatoria en tres dimensiones. 


Las condiciones y definiciones son las siguientes:
\begin{itemize}
\item La marcha se hace en coordenadas cartesianas y empieza en el origen $x_{0}=0, y_0=0, z_0=0$. 
\item Cada paso que toma la marcha siempre tiene longitud igual a uno, $r=\sqrt{d_{ix}^2 + d_{iy}^2 + d_{iz}^2}\equiv 1$ donde los $d_{i}$ indican los desplazamientos del paso $i$ en cada una de las coordenadas. 
\item La direcci\'on en que se hace el desplazamiento es aleatoria y corresponde a una distribuci\'on plana de probabilidad sobre la esfera. Es decir, no puede haber ninguna direcci\'on privilegiada.  
\item Luego de $N$ pasos la marcha se encuentra en la posici\'on $x_N, y_N, z_N$ a una distancia $r_N=\sqrt{x_{N}^2 + y_{N}^2 + z_{N}^2}$ del centro.
\end{itemize}

Esta marcha es difusiva, es decir, que luego de un $N_D$ suficientemente grande puedo esperar que la marcha se encuentre alejada del origen una distancia arbitraria $D$. El objetivo de este ejercicio es estudiar la relaci\'on estad\'istica entre $N_D$ y $D$.

Escriba un programa (c\'odigo fuente \verb"marcha_3D.c") que calcule
$M$ marchas aleatorias. El programa debe calcular el valor medio y la
dispersi\'on de los $M$ valores de $N_D$, cuando un $D$ determinado se fija.

En este caso nos interesa hacerlo para una serie de 50 valores diferentes $D=10, 20,\ldots,490,500$. 

Despu\'es de compilado el programa se debe ejecutar como
\begin{verbatim}
./a.out M
\end{verbatim}
donde $M>2$ es el n\'uemero total de marchas aleatorias.

\begin{itemize}
\item (30 puntos) El programa debe funcionar de manera correcta para valores arbitrarios de $M$. Al final de su ejecuci\'on el programa debe crear un archivo de nombre \verb"final_stats_3D_M.dat" (donde $M$ corresponde al numero de marchas) que contiene dos columnas y 50 filas. La primera columna tiene los valores de $D=10,20\ldots, 490,500$ y la segunda tiene el valor medio $\langle N_D\rangle$ para el $D$ correspondiente. 
\item Escriba en los comentarios del c\'odigo fuente si logr\'o
  encontrar alguna relaci\'on entre $D$ y $\langle N_D\rangle$ con los datos obtenidos por su programa. 
\end{itemize}



\item
{\bf Three Wise Men}

{\textit{The story of the three wise men got me wondering: What if you did walk
towards a star at a fixed speed?  What path would you trace on the
Earth? Does it converge to a fixed cycle? —N. Murdoch}}   

Answer: \url{https://what-if.xkcd.com/25/}

\end{enumerate}




\end{document}

\end{document}
