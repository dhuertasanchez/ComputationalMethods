\documentclass{article}
\title{Taller \#7. F\'isica Computacional / FISI 2025 \\Semestre
  2013-II. \\ Profesor: Jaime E. Forero Romero} 
\date{Octubre 31, 2013}
\begin{document}
\maketitle

{\bf Esta tarea debe resolverse por parejas (i.e. grupos de 2
  personas) y la respuesta debe estar en un repositorio de github con
  un commit final antes del medio d\'ia del jueves 7 de noviembre del
  2013.} 

El objetivo de este taller es resolver la ecuaci\'on de Burgers
mediante dos m\'etodos num\'ericos diferentes. 

La ecuaci\'on de Burgers es:

\begin{equation}
\frac{\partial u}{\partial t} +cu\frac{\partial u}{\partial x}=0.
\end{equation}

Esta ecuaci\'on se puede reescribir de la siguiente manera

\begin{equation}
\frac{\partial u}{\partial t} +c\frac{\partial (u^2/2)}{\partial x}=0.
\label{eq:uno}
\end{equation}.

Un posible esquema para resolver esta ecuaci\'on es uno que calcula
las derivadas como diferencias centradas, es decir, la derivadas en el
punto $i$ dependen de los valores en $i-1$ y $i+1$:

\begin{equation}
u_i^{j+1} = u_i^{j-1} - \beta\left[\frac{(u^2)_{i+1}^j - (u^2)_{i-1}^j}{2}\right]
\end{equation}

donde $u^2$ es la velocidad al cuadrado y $\beta=c/(\Delta x/\Delta t)$
es conocido como el n\'umero de Courant-Friederichs-Levy. Para que
este esquema funcione num\'ericamente se debe cumplir la condicion
$\beta<1.0$.

Un segundo esquema m\'as preciso (conocido como el m\'etodo de
Lax-Wendroff) hace uso de una expansi\'on de hasta segundo
orden de la funci\'on. En este nuevo esquema la soluci\'on est\'a dada
por: 

\begin{eqnarray}
u_i^{j+1} &=& u_i^{j} - \frac{\beta}{4}[(u^2)^{j}_{i+1} -
  (u^2)_{i-1}^j]\\ \nonumber
& + &\frac{\beta^2}{8}[(u_{i+1}^j + u_i^j)((u^2)_{i+1}^j -
  (u^2)_{i}^{j})\\\nonumber
& - &(u_i^j+u_{i-1}^j)((u^2)_i^j - (u^2)_{i-1}^j)].
\label{eq:dos}
\end{eqnarray}

\begin{enumerate}
\item Escriba un programa que resuelva la ecuaci\'on de Burgers usando
  el esquema num\'erico de la ecuaci\'on \ref{eq:uno}. 
\item Defina arrays con 200 items para las condiciones iniciales y
  para la soluci\'on.
\item Tome la condici\'on inicial como la funci\'on sinusoidal
  $3.0\sin(\pi x)$ donde $0<x<2$ y la velocidad del sonido es $c=1$.
\item Tome las condiciones de contorno fijas donde $u(x=0)=u(x=1)=0$.
\item Encuentre la soluci\'on cuando el tiempo total transcurrido es
  $T=0.15$. Defina $\Delta t$ tomando en cuenta la condici\'on de
  estabilidad $\beta<1$. 
\item Guarde todos los valores de $u$ para todas las iteraciones.
\item Grafique la condici\'on inicial y los valores de $u$ en una
  gr\'afica 3D ($x,t,u$) para ver la evoluci\'on de la soluci\'on.
\item Repita todos los pasos anteriores pero esta vez usando el
  esquema de Lax-Wendroff. Mantenga los mismos valores de $\Delta t$
  que utiliz\'o anteriormente. 
\item Comente las diferencias entre los resultados el primer m\'etodo y Lax-Wendroff.
\end{enumerate}

Todo el desarrollo debe estar en un \'unico notebook de Ipython.

\end{document}
