\documentclass{article}
\title{Taller \#7. F\'isica Computacional / FISI 2025 \\Semestre
  2013-I. \\ Profesor: Jaime E. Forero Romero} 
\date{Abril 16 2013}
\begin{document}
\maketitle

{\bf Esta tarea debe resolverse por parejas (i.e. grupos de 2
  personas) y debe estar en un repositorio de la cuenta de github de
  uno de los miembros de cada equipo con un commit final hecho antes del
  medio d\'ia del martes 23 de Abril del 2013}  

Es com\'un encontrar estrellas que presentan un comportamiento
oscilatorio. Por ejemplo, el Sol tiene oscilaciones debido a
variaciones de presi\'on (conocidas como ondas de modos-p) con per\'iodos de 5
minutos, medibles a partir de m\'etodos de imagen doppler.  Otro
ejemplo conocido son las estrellas de tipo Cefeida que tienen un
brillo variable con  periodos que pueden estar en el orden de d\'ias o
inclusos meses.

El siguiente ejercicio plantea un modelo simple para describir una
estrella pulsante en el caso donde las pulsaciones son originadas por
variaciones en la presi\'on. Para esto trabajaremos en simetr\'ia
esf\'erica donde la \'unica coordenada importante es la radial.

En este caso la aceleraci\'on radial de un elemento infinetesimal de
gas a una distancia $r$ del centro es un balance entre gravedad y
cambios de presi\'on:

\begin{equation}
\rho\frac{d^2r}{d t^2} = -\frac{GM_r\rho}{r^2} - \frac{dP}{dr}, 
\end{equation}

donde $\rho$ es la densidad local del gas, $M_{r}$ es la masa de gas
contenida en una esfera de radio $r$ y $P$ es la presi\'on del gas que
cambia a medida que nos alejamos del centro de la estrella.

Vamos a simplificar aun m\'as este problema pensando en que toda la
masa de la estrella $M$ est\'a concentrada en un punto y alrededor,
hay una capa de tenue gas de masa $m$ y radio $R$. Esto nos permite
reescribir el balance de la ecuaci\'on anterior de la siguiente forma: 

\begin{equation}
m\frac{d^2R}{d t^2} = -\frac{GM_rm}{R^2} + 4\pi R^{2}P.
\end{equation}

En este caso la velocidad de expansi\'on/contracci\'on de la
superficie de la estrella es $v = dR/dt$. 

Ahora incluimos una condici\'on inicial sobre la termodin\'amica del
sistema diciendo que la compresi\'on y la expansi\'on son
adiab\'aticas. Es decir $PR^{3\gamma}$ es constante, donde $\gamma$
representa el cociente entre los calores espec\'ificos del gas.


\begin{enumerate}
\item 
Escriba un programa en $C$ que resuelva este problema para condiciones
iniciales: $t=0$, $R=1.7\times 10^{12}$cm, $v=0$cm/s, $P=5.6\times
10^{5}$ dinas/cm$^2$. Usando intervalos de tiempo de $10^3$ segundos,
con $\gamma=5/3$. Para la masa de la estrella puede tomar $M = 1\times
10^{34}$g y para la masa de la capa de gas $m=1\times 10^{29}$g.

\item
Prepare tres gr\'aficas de $R$, $v$ y $P$ como funci\'on del tiempo
desde $t=0$ hasta $t=1.5\times 10^{6}$ segundos.

\item 
Escriba un programa en Python que encuentre el periodo de oscilaci\'on (en
d\'ias) y el radio de equilibrio (en cm) y los escriba en un archivo
llamado \verb"period_amplitude.txt".

\item 
Haga un archivo \verb"Makefile" que ejecute los tres puntos anteriores
cuando se use el comando \verb"make" en el directorio que contiene el
c\'odigo fuente.   
\end{enumerate}

En la calificaci\'on se dar\'a un 25\% a cada uno de los puntos del 1
al 3. Solamente se recibir\'an tareas que est\'en en un repositorio de
github. 

Enviar un email al monitor del curso Daniel Felipe Duarte {\tt
  df.duarte578} en {\tt uniandes.edu.co} con el subject
\verb"RESPUESTA TALLER 7 FISICA COMPUTACIONAL". En el cuerpo del texto
debe ir la direcci\'on del repositorio donde est\'a la tarea. 


\end{document}
