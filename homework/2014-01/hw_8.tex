\documentclass{article}
\title{Taller \#8. M\'etodos computacionales / FISI 2028 \\Semestre
  2014-I. \\ Profesor: Jaime E. Forero Romero} 
\date{Mayo 6, 2014}
 \usepackage{amsmath}
\begin{document}
\maketitle

{\bf Esta tarea debe resolverse por parejas (i.e. grupos de 2
  personas) y debe estar en un repositorio de la cuenta de github de
  uno de los miembros de cada equipo con un commit final hecho antes del
  medio d\'ia del jueves 15 de Mayo.  En todos los casos se debe hacer
  en python que calcule la respuesta y hagas las graficas
  correspondientes. Todo debe poder ejecutarse con un Makefile. Todos
  los archivos se encuentran en el directorio \verb"homework/hw_8/" del
  repositorio \verb"https://github.com/forero/ComputationalMethodsData".}  


\begin{enumerate}

\item {\bf Lineas de emission} (40 puntos)


Un instrumento mide un conteo de electrones como funci\'on de la
energ\'ia de los mismos.

El modelo f\'isico que explica la emisi\'on de estos electrones
predice que el conteo como funci\'on de la energ\'ia debe corresponder
a la siguiente forma funcional

\begin{equation}
N(E) = AE^{\alpha} + B\exp\left[-\left(\frac{E-E_0}{\sqrt{2}\sigma}\right)^2\right], 
\end{equation}

donde $A$, $B$, $E_0$, $\sigma$ y $\alpha$  son par\'ametros libres
que se deben encontrar a partir de dato experimentales.

Escriba un programa que implemente MCMC para encontrar los valores
m\'as probables de estos par\'ametros junto con su incertidumbre a
partir de los datos en el archivo \verb"energy_counts.dat". El
c\'odigo tambi\'en debe preparar histogramas bidimensionales de la
densidad puntos recorridos por la cadena en el plano $x-y$ donde $x$ y $y$
representan todos los pares posibles de par\'ametros. 


\item{\bf Lotka-Volterra experimental}(60 puntos)

Un grupo de bi\'ologos toma datos por casi una d\'ecada de una
poblaci\'on de presas y predadores. Los bi\'ologos intuyen que el
n\'umero de presas $x$ y el numero de predadores $y$ se describe por
un modelo del tipo Lotka-Volterra con las siguientes ecuaciones:

\begin{equation}
\frac{dx}{dt}=x(\alpha - \beta y),
\end{equation}
\begin{equation}
\frac{dy}{dt}=-y(\gamma -\delta x).
\end{equation}

donde $\alpha$, $\beta$, $\gamma$ y $\delta$ son par\'ametors libres
que se quieren buscar a partir de los datos experimentales.

Escriba un programa que implemente MCMC para encontrar los valores
m\'as probables de estos par\'ametros junto con su incertidumbre a
partir de los datos en el archivo \verb"lotka_volterra_obs.dat". El
c\'odigo tambi\'en debe preparar histogramas bidimensionales de la
densidad puntos recorridos por la cadena en el plano $x-y$ donde $x$ y $y$
representan todos los pares posibles de par\'ametros. 

\end{enumerate}

\end{document}






\begin{equation}
\frac{\partial^2 u (x,t)}{\partial x^2} =
\frac{1}{c^2}\frac{\partial^2 u}{\partial t^2}
\end{equation}

donde $c=\sqrt{T/\rho}$ es una velocidad de propagaci\'on con $T$ la
tensi\'on de la cuerda y $\rho$ su densidad.

Como condici\'on inicial se tiene que la cuerda est\'a estirada de
forma triangular, con el m\'aximo ubicado a $8/10$ de la longitud
total de la cuerda con una altura $1$. Es decir.

\begin{equation}
u(x,t=0) = 
\begin{cases}
1.25x/L & x\leq 0.8L\\
5-5x/L & x>0.8L
\end{cases}
\end{equation}

Integre esta ecuaci\'on tomando $T=0.01$, $\rho=40$ y $L=100$. 

\item {\bf Ecuaci\'on de Schroedinger dependiente del tiempo} (60 puntos)

(Ver cap\'itulo 18.6 de Landau, Paez, Bordeianu)\\

La evoluci\'on espacial y temporal de una part\'icula cu\'antica en
1D est\'a descrita por la siguiente ecuaci\'on de Schroedinger dependiente del
tiempo 

\begin{equation}
i\frac{\partial \psi(x,t)}{\partial t} =
-\frac{\partial^2\psi(x,t)}{\partial x^2} + V(x)\psi(x,t).
\end{equation}

Donde $\psi(x,t)$ es la funci\'on de onda y $V(x)$ es un potencial
externo. En esta expresi\'on onstantes como la masa de la part\'icula
y $\hbar$ se han hecho igual es a 1/2 y 1. 

La funci\'on de onda es compleja y puede descomponerse en su parte
real y compleja   

\begin{equation}
\psi(x,y) = R(x,t) + iI(x,t), 
\end{equation}
%
de esta manera se deben resolver dos ecuaciones acopladas
\begin{equation}
\frac{\partial R(x,t)}{\partial t} =
-\frac{\partial^2I(x,t)}{\partial x^2} + V(x)I(x,t).
\end{equation}

\begin{equation}
\frac{\partial I(x,t)}{\partial t} =
+\frac{\partial^2R(xt)}{\partial x^2} - V(x)R(x,t).
\end{equation}

donde la densidad de probabilidad $\rho(x,t) = |\psi(x,t)|^2$
es la cantidad de inter\'es f\'isico para describir la evoluci\'on
temporal. 

Resuelva esta ecuaci\'on para la siguiente condici\'on inicial

\begin{equation}
\psi(x,t=0) = \exp\left[ -\frac{1}{2} \left(\frac{x-5}{\sigma_0}\right)^2\right]\exp{(ik_0x)}
\end{equation}
que representa un paquete en$x=5$ con momentum $k_0=16\pi$ y
$\sigma_0=0.05$, mientras evoluci\'ona en un potencial cuadr\'atico
$V=x^2/2$. La animaci\'on debe mostrar la evoluci\'on de $\rho(x,t)$.

\end{enumerate}


\end{document}
