\documentclass{article}
\title{Taller \#7. M\'etodos computacionales / FISI 2028 \\Semestre
  2014-I. \\ Profesor: Jaime E. Forero Romero} 
\date{Abril 29, 2014}
 \usepackage{amsmath}
\begin{document}
\maketitle

{\bf Esta tarea debe resolverse por parejas (i.e. grupos de 2
  personas) y debe estar en un repositorio de la cuenta de github de
  uno de los miembros de cada equipo con un commit final hecho antes del
  medio d\'ia del viernes 9 de Mayo. En todos los casos se debe hacer
  un programa en C que resuelva la ecuaci\'on diferencial, un programa
en python que haga una animaci\'on con los resultados y un Makefile
que compile y ejecute todos los programas.}

\begin{enumerate}

\item {\bf Cuerda Vibrando} (40 puntos)

(Ver cap\'itulo 18.2 de Landau, Paez, Bordeianu)\\

Vamos a considerar una cuerda de longitud $L$ descrita por la funci\'on
$u(x,t)$ que corresponde al desplazamiento con respecto a su
posici\'on de equilibrio. En este de una perturbaci\'on inicial, la
evoluci\'on de $u$ est\'a dada por 

\begin{equation}
\frac{\partial^2 u (x,t)}{\partial x^2} =
\frac{1}{c^2}\frac{\partial^2 u}{\partial t^2}
\end{equation}

donde $c=\sqrt{T/\rho}$ es una velocidad de propagaci\'on con $T$ la
tensi\'on de la cuerda y $\rho$ su densidad.

Como condici\'on inicial se tiene que la cuerda est\'a estirada de
forma triangular, con el m\'aximo ubicado a $8/10$ de la longitud
total de la cuerda con una altura $1$. Es decir.

\begin{equation}
u(x,t=0) = 
\begin{cases}
1.25x/L & x\leq 0.8L\\
5-5x/L & x>0.8L
\end{cases}
\end{equation}

Integre esta ecuaci\'on tomando $T=0.01$, $\rho=40$ y $L=100$. 

\item {\bf Ecuaci\'on de Schroedinger dependiente del tiempo} (60 puntos)

(Ver cap\'itulo 18.6 de Landau, Paez, Bordeianu)\\

La evoluci\'on espacial y temporal de una part\'icula cu\'antica en
1D est\'a descrita por la siguiente ecuaci\'on de Schroedinger dependiente del
tiempo 

\begin{equation}
i\frac{\partial \psi(x,t)}{\partial t} =
-\frac{\partial^2\psi(x,t)}{\partial x^2} + V(x)\psi(x,t).
\end{equation}

Donde $\psi(x,t)$ es la funci\'on de onda y $V(x)$ es un potencial
externo. En esta expresi\'on onstantes como la masa de la part\'icula
y $\hbar$ se han hecho igual es a 1/2 y 1. 

La funci\'on de onda es compleja y puede descomponerse en su parte
real y compleja   

\begin{equation}
\psi(x,y) = R(x,t) + iI(x,t), 
\end{equation}
%
de esta manera se deben resolver dos ecuaciones acopladas
\begin{equation}
\frac{\partial R(x,t)}{\partial t} =
-\frac{\partial^2I(x,t)}{\partial x^2} + V(x)I(x,t).
\end{equation}

\begin{equation}
\frac{\partial I(x,t)}{\partial t} =
+\frac{\partial^2R(xt)}{\partial x^2} - V(x)R(x,t).
\end{equation}

donde la densidad de probabilidad $\rho(x,t) = |\psi(x,t)|^2$
es la cantidad de inter\'es f\'isico para describir la evoluci\'on
temporal. 

Resuelva esta ecuaci\'on para la siguiente condici\'on inicial

\begin{equation}
\psi(x,t=0) = \exp\left[ -\frac{1}{2} \left(\frac{x-5}{\sigma_0}\right)^2\right]\exp{(ik_0x)}
\end{equation}
que representa un paquete en$x=5$ con momentum $k_0=16\pi$ y
$\sigma_0=0.05$, mientras evoluci\'ona en un potencial cuadr\'atico
$V=x^2/2$. La animaci\'on debe mostrar la evoluci\'on de $\rho(x,t)$.

\end{enumerate}


\end{document}
