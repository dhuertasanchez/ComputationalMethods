\documentclass{article}
\title{Taller \#6. M\'etodos computacionales / FISI 2028 \\Semestre
  2014-I. \\ Profesor: Jaime E. Forero Romero} 
\date{Abril 8, 2014}
\begin{document}
\maketitle

{\bf Esta tarea debe resolverse por parejas (i.e. grupos de 2
  personas) y debe estar en un repositorio de la cuenta de github de
  uno de los miembros de cada equipo con un commit final hecho antes del
  medio d\'ia del martes 22 de Abril del 2013. El esqueleto de todos los c\'odigos debe estar en el repositorio antes del medio d\'ia del
  viernes 10 de Abril.}   

\begin{enumerate}

\item {\bf Din\'amica de cazadores y de presas} (20 puntos)


Tenemos dos especies, una caza a la otra. El n\'umero de presas es $x$
y el numero de cazadores es $y$. Su evoluc\'ion temporal est\'a
descrita por las siguientes dos ecuaciones 

\begin{equation}
\frac{dx (t)}{dt} = Ax(t) - Bx(t)y(t)
\end{equation}

\begin{equation}
\frac{dy (t)}{dt} = -Cy(t) + Dx(t)y(t)
\end{equation}

\begin{itemize}
\item Encuentre los valores de $x(0)$ y $y(0)$ para que exista equilibrio
($\dot{x}=\dot{y}=0$) asumiendo que $A=20$, $B=1$, $C=30$ y $D=1$.

\item Var\'ie las condiciones iniciales de la siguiente manera: $y(0)$
  se mantiene fijo en el valor encontrado en el punto anterior
  mientras que $x(0)$ disminuye en una unidad con respecto al valor
  encontrado en el punto anterior hasta llegar a 1. Solucione el
  sistema de ecuaciones  para $0<t<1$ y este conjunto de condiciones
  iniciales y represente las soluciones en el plano $x-y$. 

  El repositorio debe tener al menos tres c\'odigos con los siguientes
  nombres: \verb"volterra_lotka.c" que resuelve el sistema de ecuaciones
  diferenciales y \verb"plot_solutions.py" que prepara los datos y la
  gr\'afica de las soluciones en el plano $x-y$. 

  Todo se debe poder ejecutar con un Makefile.

\end{itemize}

\item {\bf Din\'amica de tres masas} (30 puntos)

Escriba un programa que haga una integraci\'on con un Runge-Kutta de
cuarto orden para el problema de las tres masas mencionado en
clase para un tiempo de integraci\'on de $100$ a\~nos. La \'unica
diferencia es que las velocidades iniciales van perpendiculares al
plano definitdo por las tres masas.

\item {\bf Colisiones de galaxias} (50 puntos)

Vamos a seguir el ejemplo  del paper cl\'asico de Toomre \& Toomre
\textit{Galactic Bridges and Tails} en el Astrophysical Journal, Vol. 178,
pp. 623-666 (1972), el cual se encuentra repositorio como
\verb"homework/2013-02/TT.pdf" 

Una galaxia ser\'a descrita como un (1) cuerpo central de masa $M$ con $N$ cuerpos (estrellas) en \'orbitas que la rodean. Estos cuerpos van a estar distribuidos de manera uniforme sobre un c\'irculo de radio $R$.

El movimiento {\bf de cada una} de las part\'iculas que rodean a la masa
central est\'a determinado por la siguiente ecuaci\'on diferencial
ordinaria vectorial de segundo orden: 
 
\begin{equation}
\frac{d^2\vec{r}}{dt^2} = -\frac{GM}{r^2} \hat{r}
\end{equation}

donde $\vec{r}$  es un vector que va de la masa central a la
part\'icula, $\hat{r}$ es el vector unitario correspondiente,
$r=|\vec{r}|$, $M$ es la masa del cuerpo central y $G$ es la constante
de gravitaci\'on.  Una vez se conocen las posiciones y velocidades
iniciales de cada una de las masas $\vec{r}_0$, $\vec{v}_0$ es posible
conocer la posici\'on y la velocidad en momentos siguientes. Noten que
estamos usando una  aproximaci\'on donde la fuerza que siente cada
part\'icula solamente se debe solamente a la masa central. Esto es
equivalente a decir a  que la masa de todas las part\'iculas es
despreciable con respecto a la masa central. As\'i mismo la masa central no sie te ninguna  fuerza apreciable por parte de las part\'iculas. 

\begin{enumerate}
\item Escribir un programa que genere
  las condiciones  iniciales (posiciones y velocidades) para que las
  $N$ part\'iculas de una galaxia de disco orbiten de manera estable
  en c\'irculos alrededor de la masa  central. 
  El centro de la galaxia debe ubicarse en
  el punto $x_{0}\hat{i} + y_{0}\hat{j} + z_{0}\hat{k}$ kpc y la
  velocidad del centro de masa es $v_{0x}\hat{i}+v_{0y}\hat{j} +
  v_{0z}\hat{k}$ km/s y donde $x_{0}$, $y_{0}$, $z_{0}$, $v_{0x}$,
  $v_{0y}$ y $v_{0z}$ son  n\'umeros arbitrarios que se deben como
  argumento del c\'odigo al  momento de ser ejecutado. Igualmente los
  valores de $M$, $R$ y $N$ deben ser n\'umeros que entran como
  par\'ametros de entrada al momento de la ejecuci\'on. 

  Para las cantidades de entrada las unidades son $M_{\odot}$,
  distancias en kpc y velocidades en km/s. 

  Estas condiciones iniciales (posiciones y velocidades) deben ser
  escritas en un archivo de texto al momento de ejecutar el
  c\'odigo. En este archivo la primera columna ser\'a un entero que
  llamaremos ID y corresponde a un n\'umero entero de cada
  part\'icula. El ID es negativo si se trata de la part\'icula
  central, y positivo si se trata de las part\'iculas en las \'orbitas
  circulares. 

\item 
  Escribir el c\'odigo que evolucione la posici\'on y la velocidad de
  cada una de las part\'iculas durante un tiempo $T$ (medido en
  unidades de miles de millones de a\~nos).  Este c\'odigo debe leer
  las condiciones iniciales generadas por el c\'odigo anterior. Se
  debe utilizar un m\'etodo de Runge-Kutta de cuarto orden para
  integrar la ecuaci\'on de movimiento para cada part\'icula del
  disco.  El c\'odigo debe generar cinco archivos 
  de texto con el entero de identidad, posiciones, velocidades de las
  part\'iculas en $M$ momentos diferentes equiespaciados en el tiempo
  $T$ de evoluci\'on del sistema
  evoluci\'on del sistema. El c\'odigo debe tomar como argumento el
  nombre del archivo con las condiciones iniciales, el tiempo $T$ y el
  n\'umero de momentos $M$. Cada vez que se escriba un archivo se debe
  usar el mismo formato utilizado para las condiciones iniciales.

\item 
  Prepare un programa de python que toma como entrada un n\'umero arbitario de 
  archivos generados por el c\'odigo anterior y prepara gr\'aficas que
  muestran las posiciones.  Utilice estre programa para mostrar que
  las part\'iculas que representan el disco de la galaxia siguen en su
  \'orbita circular despu\'es de 5 mil millones de a\~nos. 
  
\item 
  Este punto incluye una segunda galaxia id\'entica a la
  primera. Si se considera que la galaxia anterior tiene un centro de
  masa en la posici\'on $(0\hat{i}+0\hat{j})$ kpc y una velocidad de centro de
  masa nula, vamos a considerar ahora que la masa central de la
  segunda galaxia tiene una posicion $(150\hat{i}+200\hat{j})$ kpc y
  el centro de masa  tiene una velocidad inicial $-100\hat{j}$ km/s. La
  direcci\'on $\hat{z}$ es perpendicular al plano del disco. 

  Las condiciones iniciales de cada una de las galaxias son generadas
  con el c\'odigo del primer punto con $N=10^5$ y
  $R=20$kpc. Concatenando los dos archivos (utilizando el comando
  \verb"cat") debe generar un archivo \'unico de condiciones iniciales.
  
  Ahora evolucione estas nuevas condiciones iniciales por un intervalo
  $T=5$  para ver la interacci\'on de las dos galaxias.  Prepare
  gr\'aficas de la posicion de las part\'iculas de las dos galaxias en
  $M=5$ momentos diferentes equiespaciados en el tiempo $T$.

  Note que en esta configuraci\'on las masas centrales sienten su
  influencia mutua y su \'orbita tambi\'en debe ser calculada.

\end{enumerate}

El repositorio debe tener al menos tres c\'odigos con los siguientes nombres:

\begin{enumerate}
\item \verb"IC.c": c\'odigo que genera las condiciones iniciales de
  una sola galaxia con las posiciones y velocidades del centro de masa
  dadas como par\'ametros de entrada al momento de ejecutar el programa.
\item \verb"evolve.c": c\'odigo que evoluciona un archivo con 
  condiciones iniciales dadas al momento de ejecutar el programa.
\item \verb"plots.py": c\'odigo que prepara gr\'aficas de
  posiciones, tomando como entrada un n\'umero
  arbitrario de archivos de ID, posiciones y velocidades.
\end{enumerate}

Todos los pasos deben poder ejecutarse con un Makefile.
Inspiraci\'on: \verb"http://viz.adrian.pw/galaxy/"

\item {\bf Colapso gravitacional} (10/20/30 puntos)

Escriba un programa en C que integre las ecuaciones de movimiento de
$N=10^4$ masas iguales (cada una de $1$M$_\odot$) que se encuentran
distribuidas inicialmente de manera aleatoria dentro de una esfera de
radio $20$ parsec. Modifique las ecuaciones de Newton para que la
fuerza de gravedad sea una constante si la separaci\'on entre dos
masas es menor que $0.01$ parsec. Se debe describir la evoluci\'on del
sistema por $10^9$ a\~nos de tal manera que la energ\'ia total se
conserve dentro de una incertidumbre menor a $\Delta E/|E|<10^{-5}$.  

Prepare gr\'aficas de la evoluc\'ion del sistema en 20 momentos
diferentes, as\'i como de la cantidad $G=2K+U$ en cada paso de tiempo,
donde $K$ es la energ\'ia cin\'etica y $U$ es la energ\'ia potencial.  
 
El repositorio debe tener al menos dos c\'odigos. Uno llamado
\verb"nbody.c" que resuelve las ecuaciones diferenciales y otro
\verb"plot_nbody.py" que produce las gr\'aficas necesarias.

Todo se debe poder ejecutar usando un Makefile.

\end{enumerate}


\end{document}
