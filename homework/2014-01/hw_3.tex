\documentclass{article}
\textheight=25.5cm
\textwidth=16.0cm
\oddsidemargin=-0.5cm
\topmargin=-3.0cm
\usepackage[pdftex]{graphicx}
\usepackage[utf8]{inputenc}
\usepackage[spanish]{babel}
\title{Taller \#3 de M\'etodos Computacionales\\ FISI 2028, Semestre 2014 - 10}
\author{Profesor: Jaime Forero}
\date{Mi\'ercoles 19 de Febrero, 2014}
\begin{document}
\maketitle
\thispagestyle{empty}


{\bf Importante}
\begin{itemize}

\item Los cinco archivos con el c\'odigo en Python que soluciona esta
  tarea deben subirse a trav\'es de sicuaplus antes de las 4PM del
  jueves 27 de Febrero como un \'unico archivo zip con el nombre
  \verb"NombreApellidos_hw3.zip", por ejemplo yo deber\'ia subir un
  archivo llamado \verb"JaimeForero_hw3.zip". 

\item La nota m\'axima de este taller es de 100 puntos. Los puntos indicados
  en cada literal solamente se otorgan si el programa compila y da los
  resultados esperados seg\'un la descripci\'on de cada punto.
 
\end{itemize}

\begin{enumerate}
\item
{\bf Conjunto de mandelbrot}\\
(20 puntos)
Escriba un programa que prepare en una grid de resolucion de
$1024\times 1024$ pixeles las primeras $n$ iteraciones del
Conjunto de
Mandelbrot. Ver: \verb"http://en.wikipedia.org/wiki/Mandelbrot_set". 

El programa debe poder ejecutarse como

\begin{verbatim}
python plot_mandelbrot.py n
\end{verbatim}
Donde $n$ es el n\'umero de iteraciones. El programa debe producir un
archivo en pdf llamado \verb"mandelbrot_n.pdf" para guardar la
gr\'afica. 

\item 
{\bf Distancias sobre la esfera}\\
(20 puntos)
Escriba un programa que calcula la distancia en kil\'ometros entre dos pares de
ciudades dada su longitud y latitud.

El programa debe poder ejecutarse como
\begin{verbatim}
python ditancia_esfera.py -144.0 +10.0 +20.0 -5.0
\end{verbatim}

Donde los dos primeros valores corresponden a la latitud y longitud de
la primera ciudad y los dos siguientes a la latitud y
longitud de la segunda ciudad. Estas coordenadas deben estar en
grados. La distancia debe imprimirse en pantalla.

\item
{\bf Movimiento Browniano}\\

Vamos a estudiar un caso de movimiento browniando simplificado donde
una part\'icula se mueve en l\'inea recta por una distancia de $1$
antes de cambiar aleatoriamente su direcci\'on. Este caso simplificado
puede ayudarnos a describir diferentes problemas f\'isicos desde la
difusi\'on de fotones en el interior del Sol o la propagaci\'on de
fotones resonantes en una galaxias.

\begin{itemize}
\item[a)] (15 puntos) Escriba un programa en Python que describa el movimiento
  browniano de una part\'icula que comienza en el origen del sistema
  de coordenadas. El programa debe preparar una gr\'afica de la distancia
  hasta el origen como funci\'on del n\'umero de pasos. 
El programa debe poder ejecutarse como
\begin{verbatim}
python browniano_2D.py n
\end{verbatim}
Donde $n$ es el n\'umero m\'aximo de iteraciones. La gr\'afica final
debe llamarse \verb"browniano_2D_n.pdf".

\item[b)] (20 puntos) Ahora consideremos un caso simplificado de la difusi\'on de fotones en el interior solar. Los fotones producidos
  en el centro de la distribuci\'on de gas hacen un movimiento
  browniano de paso $1$, para una esfera de tama\~no $10^5$. Considere
  ahora que tiene $n$ fotones. Calcule un histograma con los n\'umero
  de pasos que le toma a los fotones escapar de la distribuci\'on de gas.

  El programa debe poder ejecutarse como
\begin{verbatim}
python difusion_solar_central.py n
\end{verbatim}
Donde $n$ es el n\'umero de fotones a propagar. 


El histograma final
debe llamarse \verb"histo_difusion_solar_central_n.pdf".

\item[c)] (25 puntos) Considere ahora el mismo caso anterior pero esta vez los $n$ fotones est\'an distribuidos inicialmente de manera homog\'enea en
  la esfera.

El programa debe poder ejecutarse como

\begin{verbatim}
python difusion_solar_homogenea.py n
\end{verbatim}

Donde $n$ es el n\'umero de fotones a propagar. 

El histograma final
debe llamarse \verb"histo_difusion_solar_homogenea_n.pdf".

\end{itemize}

\end{enumerate}

\end{document}
