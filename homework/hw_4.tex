\documentclass{article}
\title{Taller \#4. F\'isica Computacional / FISI 2025 \\Semestre
  2013-I. \\ Profesor: Jaime E. Forero Romero} 
\date{Febrero 21 2013}
\begin{document}
\maketitle

{\bf Esta tarea debe resolverse por parejas (i.e. grupos de 2
  personas) y debe estar en un repositorio de la cuenta de github de
  uno de los miembros de cada equipo con un commit final hecho antes del
  medio d\'ia del jueves 7 de Marzo del 2013}  

Es el a\~no 2150. Despu\'es de los descubrimientos de exoplanetas
similares a la Tierra en el 2020 y de haber resuelto en el 2080 el
problema de viaje inter-estelar en escalas de tiempo humanas, es una
pr\'actica com\'un para los estudiantes de la Universidad de los Andes
hacer salidas de campo a otros planetas.
 
En una de estas salidas de campo, el objetivo de los estudiantes de
las 300 secciones de Fisica I (la Universidad ahora cuenta con 500 mil
estudiantes) es repartirse sobre un meridiano del planeta Brahe-$314$
para hacer experimentos de movimiento parab\'olico y deducir el valor
de la gravedad en diferentes lugares del planeta.

Sus experimentos de alt\'isima precisi\'on (con errores en mediciones
de tiempos y posiciones despreciables) en c\'amaras gigantes de alto
vac\'io consisten en hacer 6 tiros parab\'olicos y medir durante 4
segundos la trayectoria del proyectil. 

Para evitar sesgos en las mediciones, cada uno de los tiros
parab\'olicos en cada una de las 300 posiciones sobre el meridiano de
Brahe-$314$ tiene diferentes velocidades iniciales. 

Dentro del repositorio en \verb"homework/hw4_data/" hay 1800
archivos diferentes con los datos de estos experimentos,
correspondientes a 6 tiros parab\'olicos en 300 posiciones. Los
archivos tienen un nombre del tipo:
\begin{center}
\verb"experiment_ID_3_theta_180.0.dat"\\
\end{center}

Eso indica que ese es el experimento con indentificaci\'on 3 (la
cuenta para la identificaci\'on empieza en 0) hecho al \'angulo
$180.0$ polar sobre el meridiano (en analog\'ia con un sistema de
coordenadas esf\'ericas). En este caso el archivo de datos
corresponder\'ia a los datos tomados en el polo sur geogr\'afico. 

El objetivo de la tarea es escribir un c\'odigo en Python (o en un
notebook the IPython) que haga las siguientes tareas:
\begin{enumerate}

\item
Por cada uno de los archivos de datos devuelva los par\'ametros
$g,v_{0x},v_{0y}$, que corresponden a la aceleraci\'on de la gravedad
en ese sitio, la velocidad inicial en la direcci\'on perpendicular a
la gravedad y la velocidad inicial en la direccion paralela a la gravedad.

\item
Procesa los datos anteriores a trav\'es de Principal Component
Analysis para verificar si los valores obtenidos para la gravedad son
independientes de los valores iniciales de las velocidades iniciales.  


\item
Prepara una gr\'afica de los valores medios de la gravedad como funci\'on del
\'angulo polar $\theta$.  

\item 
El programa debe preparar una lista de las variaciones de la gravedad
parametrizada por 

\begin{displaymath}
F=1.0-\frac{g_{\rm media}}{9.81\mathrm{m\ s}^{-2}},
\end{displaymath}

en funci\'on del \'angulo polar $\theta$.

\item 
Las variaciones de F corresponden a las fluctuaciones en la gravedad
respecto al valor de referencia terrestre. Si el programa logra ver
estas fluctuaciones ahora intente hacer un ajuste a esta variaci\'on
con una funci\'on del tipo $F = F_{0}\sin(\theta) + c_{0}$, donde
$F_{0}$ y $c_{0}$ son constantes. El programa deber preparar una
gr\'afica de los residuos de los valores observados con respecto al
mejor fit obtenido para esta funci\'on.    

\item
Enviar un email al monitor del curso Daniel Felipe Duarte {\tt
  df.duarte578} en {\tt uniandes.edu.co} con el subject
\verb"RESPUESTA TALLER 4 FISICA COMPUTACIONAL". En el cuerpo del texto
debe ir la direcci\'on del repositorio donde est\'a la tarea. 


\end{enumerate}

En la calificaci\'on se dar\'a un 20\% a cada uno de los puntos del
1 al 5.

\vspace{1cm}

Para ver una aplicaci\'on real de cartograf\'ia de la Luna a
partir de mediciones de precision de fluctuaciones del campo
gravitacional, pueden ir aqu\'i:

\verb"http://www.nasa.gov/mission_pages/grail/news/grail20121205.html"

\end{document}
